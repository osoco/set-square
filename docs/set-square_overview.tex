% ====================================================================
%    ___ ___ ___  __ ___ 
%   / _ (_-</ _ \/ _/ _ \
%   \___/__/\___/\__\___/
%
%   ~ La Empresa de los Programadores Profesionales ~
%
%
%  | http://www.osoco.es
%  |
%  | Edificio Moma Lofts
%  | Planta 3, Loft 18
%  | Ctra. Mostoles-Villaviciosa, Km 0,2
%  | Mostoles, Madrid 28935 España
%
% ====================================================================
%
% Copyleft 2015 
%
% by Jose San Leandro
%
%%%%%%%%%%%%%%%%%%%%%%%%%%%%%%%%%%%%%%%%%%%%%%%%%%%%%%%%%%%%%%%%%%%%%%
\documentclass{beamer}

\usetheme{osoco2012}

\usepackage[english]{babel}
\usepackage[utf8]{inputenc}
\usepackage{minted}
\usepackage{listings}

% Metadata
%%%%%%%%%%%%%%%%%%%%%%%%%%%%%%%%%%%%%%%%%%%%%%%%%%%%%%%%%%%%%%%%%%%%%%

\title{set-square}
\subtitle{Building Docker images your own way}
\author{Jose San Leandro \href{http://twitter.com/rydnr}{@rydnr}}
\institute[OSOCO]{%
  \href{http://www.osoco.es}{OSOCO}
}
\date[11/2015]{November 2015}
\subject{set-square}
\keywords{set-square, phusion, docker, bash}

% Contents
%%%%%%%%%%%%%%%%%%%%%%%%%%%%%%%%%%%%%%%%%%%%%%%%%%%%%%%%%%%%%%%%%%%%%%

\begin{document}

\begin{frame}[plain]
  \titlepage
\end{frame}

\begin{frame}[plain]{Contents}
  \tableofcontents[hideallsubsections]
\end{frame}

\section{Introduction}

\subsection{Motivation}

\begin{frame}{Dockerfiles (I)}
		\begin{exampleblock}{In theory}
			\begin{itemize}
        \item Automates building Docker images.
        \item Focuses on readability.
        \item Uses a caching mechanism to speed up the process.
        \item Advocates for deterministic/idempotent recipes: the same Dockerfile produces equivalent images.
			\end{itemize}
		\end{exampleblock}
\end{frame}

\begin{frame}{Dockerfiles (II)}
  \begin{block}{\href{https://github.com/kstaken/dockerfile-examples/blob/master/apache/Dockerfile}{kstaken's} Apache dockerfile}
    \inputminted[fontsize=\small]{bash}{apache.dockerfile}
  \end{block}
\end{frame}

\begin{frame}{Dockerfiles (III)}
		\begin{alertblock}{In practice}
			\begin{itemize}
        \item It's \textbf{not a language}, only a variable-less DSL.
        \item Using the operating system's package manager makes \textbf{idempotence mostly unachievable}.
        \item Not using the operating system's package manager \textbf{complicates} Dockerfiles.
        \item Images tend to \textbf{grow so much} the Dockerfiles need to
          join steps together at the expense of simplicity.
			\end{itemize}
		\end{alertblock}
\end{frame}

\section{Features}

\begin{frame}{Features}
  \begin{block}{set-square}
    \begin{enumerate}
    \item Adds support for simple \textbf{variables} in Dockerfiles.
    \item It's \textbf{not} a template engine (it's based on \texttt{envsubst}).
    \item Maintains Dockerfile's \textit{original readability}.
    \item \textbf{Simplifies} the process of building images.
    \end{enumerate}
  \end{block}
\end{frame}

\begin{frame}{Example}
  \begin{block}{\texttt{gradle/Dockerfile.template}}
    \inputminted[fontsize=\small]{bash}{gradle.dockerfile}
  \end{block}
\end{frame}
  
\begin{frame}{Example: default values}
  By default, we'd be interested in the latest stable version.
  \begin{block}{\texttt{gradle/build-settings.sh}}
    \inputminted[fontsize=\small]{bash}{gradle.settings}
  \end{block}
  \pause

  Of course, we can customize the default values.
  \begin{exampleblock}{\texttt{gradle/.build-settings.sh}}
    \inputminted[fontsize=\small]{bash}{gradle.settings.inc}
  \end{exampleblock}
  \pause

  Even further!
  \begin{alertblock}{}
    \inputminted[fontsize=\small]{bash}{gradle.commandline}
  \end{alertblock}
\end{frame}

\begin{frame}{Common variables}
  There're some variables we'll want to override, globally:
  \begin{alertblock}{\texttt{.build.inc.sh}}
    \inputminted[fontsize=\small]{bash}{build.inc.sh}
  \end{alertblock}
\end{frame}

\section{Build process}

\begin{frame}{Build process (I)}
  \begin{block}{}
    \inputminted[fontsize=\small]{bash}{liquibase.mkdir.bash}
  \end{block}
  \pause

  \begin{exampleblock}{\texttt{liquibase/Dockerfile.template}}
    \inputminted[fontsize=\small]{bash}{dockerfile.boilerplate}
  \end{exampleblock}

\end{frame}

\begin{frame}{Build process (II)}
  \begin{block}{}
    \inputminted[fontsize=\small]{bash}{liquibase.build.bash}
  \end{block}
  \pause

  \begin{exampleblock}{}
    \inputminted[fontsize=\tiny]{bash}{liquibase.build.output}
  \end{exampleblock}
\end{frame}

\begin{frame}{Build process (III)}
  Build the image, run it, test it, fix it.
  \begin{block}{\texttt{liquibase/Dockerfile.template}}
    \inputminted[fontsize=\tiny]{bash}{liquibase.dockerfile}
  \end{block}
\end{frame}

\begin{frame}{Build process (IV)}
  You can use templates for other files too.
  \begin{exampleblock}{\texttt{liquibase/liquibase.sh.template}}
    \inputminted[fontsize=\small]{bash}{liquibase.sh}
  \end{exampleblock}

\end{frame}

\section{Conclusion}

\begin{frame}{Conclusion (I)}
  \begin{exampleblock}{Simple but useful}
    \begin{itemize}
      \item Simple and straight-forward.
      \item No hidden magic.
      \item Avoids changing values manually.
      \item Useful for CI jobs.
    \end{itemize}
  \end{exampleblock}

  \pause

  \begin{alertblock}{Try it}
    \href{https://github.com/rydnr/set-square}{https://github.com/rydnr/set-square}
  \end{alertblock}

\end{frame}

\begin{frame}{Conclusion (II)}
  \begin{exampleblock}{\href{https://github.com/rydnr/set-square-phusion-images}{set-square-phusion-images}}
    \begin{itemize}
      \item Catalog of Docker images.
      \item Based on \href{https://github.com/phusion/baseimage-docker}{phusion/baseimage-docker}.
      \item Out-of-the-box (but optional):
        \begin{itemize}
          \item monitoring,
          \item backup,
          \item logging.
        \end{itemize}
    \end{itemize}
  \end{exampleblock}

\end{frame}

\begin{frame}{Titulo}
  \begin{block}{cabecera}
    \begin{itemize}
    \item Lorem Ipsum

    \end{itemize}
  \end{block}

\end{frame}

\end{document}
